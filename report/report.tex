% !TEX TS-program = pdflatex
% !TEX encoding = UTF-8 Unicode

% This is a simple template for a LaTeX document using the "article" class.
% See "book", "report", "letter" for other types of document.

\documentclass[11pt]{article} % use larger type; default would be 10pt

\usepackage[utf8]{inputenc} % set input encoding (not needed with XeLaTeX)

%%% Examples of Article customizations
% These packages are optional, depending whether you want the features they provide.
% See the LaTeX Companion or other references for full information.

%%% PAGE DIMENSIONS
\usepackage[a4paper,left=2cm,right=2cm,top=2cm,bottom=2cm]{geometry}
%\usepackage{geometry} % to change the page dimensions
% \geometry{a4paper} % or letterpaper (US) or a5paper or....
% \geometry{margin=0in} % for example, change the margins to 2 inches all round
% \geometry{landscape} % set up the page for landscape
%   read geometry.pdf for detailed page layout information

\usepackage{graphicx} % support the \includegraphics command and options

\usepackage[parfill]{parskip} % Activate to begin paragraphs with an empty line rather than an indent

%%% PACKAGES
\usepackage{booktabs} % for much better looking tables
\usepackage{array} % for better arrays (eg matrices) in maths
\usepackage{paralist} % very flexible & customisable lists (eg. enumerate/itemize, etc.)
\usepackage{verbatim} % adds environment for commenting out blocks of text & for better verbatim
\usepackage{subfig} % make it possible to include more than one captioned figure/table in a single float
\usepackage{amsmath}
\usepackage{amssymb}
\usepackage{logicproof}
\usepackage{tikz}
\usepackage{hyperref}
\usetikzlibrary{arrows,petri,topaths}
\usepackage{float}
\usepackage{graphicx}
\usepackage[T1]{fontenc}
\usepackage{listings}
\usepackage{pdflscape}
\lstset{
  basicstyle=\ttfamily,
  mathescape
}	
% These packages are all incorporated in the memoir class to one degree or another...

%%% HEADERS & FOOTERS
\usepackage{fancyhdr} % This should be set AFTER setting up the page geometry
\pagestyle{fancy} % options: empty , plain , fancy
\renewcommand{\headrulewidth}{0pt} % customise the layout...
\lhead{}\chead{}\rhead{}
\lfoot{}\cfoot{\sffamily\thepage\normalfont}\rfoot{}

%%% SECTION TITLE APPEARANCE
\usepackage{sectsty}
\allsectionsfont{\sffamily\mdseries\upshape} % (See the fntguide.pdf for font help)
% (This matches ConTeXt defaults)

%%% ToC (table of contents) APPEARANCE
\usepackage[nottoc,notlof,notlot]{tocbibind} % Put the bibliography in the ToC
\usepackage[titles,subfigure]{tocloft} % Alter the style of the Table of Contents
\renewcommand{\cftsecfont}{\rmfamily\mdseries\upshape}
\renewcommand{\cftsecpagefont}{\rmfamily\mdseries\upshape} % No bold!
\newcommand{\qedsymbol}{\rightline{$\blacksquare$}}
\renewcommand{\familydefault}{\sfdefault}
\renewcommand{\thesection}{\hspace{-0.5cm}\arabic{section}}
\renewcommand{\thesubsection}{\alph{subsection})}
\renewcommand{\thesubsubsection}{}

\usepackage[style=authoryear]{biblatex}
\addbibresource{library.bib}

%%% END Article customizations

%%% The "real" document content comes below...

\title{\vspace{-1.6cm}Computational Modelling in the Humanities and Social Sciences Assignment \\
	\vspace{0.5cm}\large Measuring city accessibility using OpenStreetMap data\vspace{-0.3cm}}
\author{zrlr73}
\date{} % Activate to display a given date or no date (if empty),
         % otherwise the current date is printed 

\begin{document}
\maketitle

\section{Introduction}

As an occasional OpenStreetMap (OSM) contributor, I decided that using data directly from OSM in my assignment would be interesting. When a friend told me about a problem she had read about in \textit{Invisible Women} by Caroline Criado Pérez \citeyear{Perez2019}, I decided to try and produce a tool that is applicable to this issue.

The problem is that cities are designed by commuting men, and therefore can end up optimised for this subset of the population, to the detriment of others. The book specifically discusses women, who have different travel patterns to men due to their (usually) increased care responsibilities and different way of life.

For example, many cities feature a 'radial' transit network: designed with links that lead from the suburbs into the city centre, with infrastructure for transferring between these transit lines in the centre. Unfortunately, this system often fails to serve those who do not commute into the city centre for their work on a daily basis. Those who, for example, only need to travel among the suburbs, find themselves facing unreasonable journey times.

Similarly, cities are often designed for cars. With many households owning one or no cars, and that car usually being primarily used by a man, this disproportionately benefits men - not to mention the fact that, thanks to space constraints, more developed roads often worsen the experience for pedestrians. Pavements may be partially blocked with street furniture such as street lights and parking machines, or otherwise too narrow for buggies or for people with shopping to pass each other comfortably. As it is usually women who perform errands such as taking children out of the house or going to the shops, they are the ones who bear most of the brunt for these poor design choices.

Clearly, I am unable to directly change these problems. However, I thought it would be interesting to attempt to create a tool that, given an OpenStreetMap file covering a city, will score that city on its accessibility (primarily, its accessibility to women). Given that OSM is a Volunteered Geographic Information (VGI) project, this will be complicated by the fact that I cannot rely on the data to be accurate, current or even present (with the last being the most likely scenario for many cities). Therefore I will need to attempt to utilise metrics that are (relatively) unaffected by lack of correct data.

There is also no guarantee that the provided data will perfectly encompass a city - indeed, when exporting my custom datasets as described below, I simply used a rectangle roughly covering the relevant metropolitan area. In general, this means I will need to make use of relative information (eg, ratio of feature $x$ against feature $y$) instead of absolute information (eg, density of feature $x$ per square kilometre).


\section{Tools Utilised}
For this project, I have employed a number of tools. These have primarily been on the map import side, and are as follows:

\subsection{Python 3.9.2}
I am performing all of the required processing and analysis in Python, as it is well-supported by the open-source community, is accessible and popular, and performs well. Version 3.9.2 was used as it was already installed on my computer and is only three patches behind the latest version.

\subsection{BBBike Download Server}
This program, available at \href{https://download.bbbike.org/osm/}{https://download.bbbike.org/osm/}, allows users to download extracts of the OSM map. While it does feature pre-extracted files for many urban areas, I also chose to extract custom data for a number of smaller areas, including Durham, Newcastle, and Southfields (an area of south-west London in which I used to live). These files were all downloaded in Protocolbuffer Binary Format (PBF; \texttt{.osm.pbf}) file format.

\subsection{PyDriosm}
PyDriosm is an open-source library which provides features for the parsing and storage of OSM data. Its main use in my software is to parse downloaded PBF files with the \href{https://pydriosm.readthedocs.io/en/latest/_generated/pydriosm.reader.parse_osm_pbf.html}{\texttt{pydriosm.reader.parse\_osm\_pbf()}} function, returning the data as a Pandas DataFrame. PyDriosm can be installed using \texttt{pip}, with the package name \texttt{pydriosm}.

When installing on Windows 10 / Python 3.9.2, I had some trouble building the \texttt{GDAL} dependency - resolving this required installing Visual C++ build tools from Microsoft and then manually installing \texttt{GDAL} itself using a wheel file. More information on this can be found in the \href{https://pydriosm.readthedocs.io/en/latest/installation.html}{PyDriosm documentation}.

\subsection{tqdm}
\href{https://tqdm.github.io/}{\texttt{tqdm}} is a lightweight Python module for adding progress bars to iterative processes. I have employed it here to monitor progress on analysis steps.

\subsection{Pickle?}

\subsection{Shapely?}


\section{Implementation}
One major issue that was encountered was the very large times required to load PBF files into usable Python structures. Therefore, I started by creating a wrapper function, \texttt{ParsePBF()}. This function, upon reading a new PBF file, would store the parsed structure to disk in a "\texttt{cache}" folder using Python's inbuilt \href{https://docs.python.org/3/library/pickle.html}{\texttt{pickle}} library. Upon being asked to load the same file again, it would then check for the existence of an existing cache file before attempting to load it from the PBF. Loading these pickle files takes on the order of seconds, whereas large PBF map dumps have taken up to an hour to load in my experience (using an Intel Core i5-6300HQ).

Methods that I came up with to measure a city's accessibility are as follows:

Firstly, I sought to address the problem of transit system design. My idea was to find the number of 'routes' (eg, train lines, bus routes) that each station / stop was on, for every station / stop, and then find the standard deviation across the city. A large standard deviation would indicate the presence of both many highly-connected stations and many poorly-connected stations, which would be most likely caused by a radial structure with poorly-connected suburban transport links and a strongly-connected 'downtown' area. Conversely, a low standard deviation would indicate even distribution of transport links across the city, and therefore increased accessibility for those that do not commute into the city to work.

Unfortunately, upon attempting to access this data, I found that the majority of stations simply are not tagged with any information on which route(s) they are on. Some examples of urban metro networks do store limited line information, but these are the exception rather than the norm. Indeed, from my limited testing of various cities, London was the only one which stored a list of lines with each station, and even that was only for Underground lines. Similarly, bus stops also are not tagged with the bus routes that stop there.

I believe that the problem here is that OSM data is designed to provide information to be used to create a graphical map - and more nuanced data on how stations relate to one another is simply not required. Given more time, I would try to cross-reference railway line data with station co-ordinates, or use alternative data sources such as \href{https://www.openrailwaymap.org/}{OpenRailwayMap}, but this was unfortunately not feasible for this project's timeframe. For the time being, I have implemented it anyway, assigning stations with no tagged lines a number of lines of 1, but also provide the percentage of stations with no tagged lines in order to allow the user to judge the reliability of the metric.

I also had the idea of measuring the presence of bike share schemes (for example, London's \href{https://tfl.gov.uk/modes/cycling/santander-cycles}{Santander Cycles} system), but articles I found on these appear to report that their benefits for non-commuters are inconclusive because they are typically optimised for city-centre commuter use rather than more local suburban journeys \parencite{Beecham2014}. Since the level of benefit for non-commuters versus commuters is not agreed upon, I decided to omit processing them.

One method with which I did have some success was attempting to measure the level of zoning. While areas of cities are traditionally allocated specific purposes (residential, retail, industrial etc), in a practice known as single-use zoning, \cite{Perez2019} points out that this method of planning hugely hinders those who don't have access to a car, or live in areas with poorer public transport links. Cities like Vienna have had more success with mixed-use zoning: the design of cities where all residential areas are constructed alongside amenities such as schools, playgrounds, doctors' surgeries and shops. This ensures that the majority of services can be found within walking distance of a house, and also increases the perceived safety of residential areas through improved lighting and increased night-time activity.

Therefore, I decided to measure the average distance from a house to the nearest instance of each of these services. In order to avoid exhaustively checking the distance to, for example, every school in the city, for each house, I came up with an idea: finding a school that's somewhat near the given house through sampling and then limiting the search to only schools that have a longitude within that distance of the house's longitude. This worked well, removing complexity by well over half in my observations.



\section{Evaluation}

\section{Conclusions}

\section{References?}
\printbibliography

\end{document}
